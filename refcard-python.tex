\documentclass{article}

\usepackage[a4paper, landscape, margin=1cm]{geometry}
\usepackage{fontspec}
\usepackage[french]{babel}
\usepackage[fontsize=6.5pt]{scrextend}
\usepackage[T1]{fontenc}
\usepackage{multicol}
\usepackage{tabularx}
\usepackage{sectsty}
\usepackage{lmodern}
\usepackage{stix}
\usepackage{listings}
\usepackage{xcolor}
\usepackage{multirow}
\usepackage{titlesec}

\input{revision}

\setlength{\parskip}{0.2em}
\setlength{\parindent}{0em}

% Highlight configuration for C programming language
\lstset{
  language=python,
  breaklines=true,
  keywordstyle=\bfseries\color{black},
  basicstyle=\ttfamily\color{black},
  emphstyle={\em \color{gray}},
  emph={expr, type, NAME, ptr, name, expr, value, filename, label, member, type},
  mathescape=true,
  keepspaces=true,
  showspaces=false,
  showtabs=true,
  tabsize=3,
  columns=fullflexible,
  escapeinside={(*}{*)}
}

% Configuration
\renewcommand{\familydefault}{\sfdefault}

\sectionfont{\fontsize{12}{15}\selectfont}
\subsectionfont{\fontsize{10}{12}\selectfont}

\allsectionsfont{\sffamily\underline}

% No pages numbering
\pagenumbering{gobble}

% Titles and paragraphs more compact
\titlespacing*{\section}{0pt}{0pt}{0pt}
\titlespacing*{\subsection}{0pt}{0pt}{0pt}

\newlength\mybaselinestretch
\mybaselinestretch=0pt plus 0.02pt\relax
\addtolength{\baselineskip}{\mybaselinestretch}

\setlength\parindent{0pt}
\setlength\tabcolsep{1.5pt}
\setlength{\columnseprule}{0.4pt}

% Macros
\newcommand{\tab}{\hspace{2em}}
\newcommand{\etc}{\small \ldots}
\newcommand{\any}{$\hzigzag$~}
\newcommand{\spc}{$\mathvisiblespace$}
\newcommand{\cd}{\lstinline}

\begin{document}

\begin{multicols*}{3}

\begin{center}
  {\Large \bf Carte de référence Python 3.x} \\
  HEIG-VD -- version \revision \ -- \revisiondate \\
\end{center}

Cette carte de référence peut être utilisée durant les travaux écrits
des cours utilisant \emph{python} à moins que le contraire soit explicitement formulé.
Elle est une liste non exhaustive des possibilités du langage Python 3.

Ce travail est inspiré de \emph{Learn X in Y}.

Signification des termes utilisés dans cette carte de référence.

\begin{tabularx}{\linewidth}{
  >{\hsize=0.5\hsize}X% 10% of 4\hsize
  >{\hsize=1.5\hsize}X% 30% of 4\hsize
  >{\hsize=0.5\hsize}X% 30% of 4\hsize
  >{\hsize=1.5\hsize}X% 30% of 4\hsize
     % sum=4.0\hsize for 4 columns
  }

  \tt \etc      & Continuation logique    & \tt \any    & N'importe quoi d'accepté \\
  \tt /\any/    & Expression régulière    & \tt \spc    & Espace obligatoire \\
  \cd{type}     & \tt int, long, float, ... & \cd{name} & \tt /[A-Za-z][A-Za-z0-9\_]+/ \\
  \cd{value}    & Valeur & \cd{NAME} & \tt /[A-Z][A-Z0-9\_]+/ \\
  \cd{filename} & Chemin de fichier relatif & \cd{expr}   & e.g. \tt a + b \\
\end{tabularx}
\hrule

\section*{Différences entre Python 2.x et Python 3.x}

\begin{lstlisting}
print('hello') print 'hello'
2/3 == 0.66    2/3 = 0
"unicode"      u"unicode"
range(1,2)     xrange(1,2)
raise Exception("error") raise "error"
\end{lstlisting}

\section*{Opérateurs et généralités}

\begin{lstlisting}
# Opérateurs 
5.0 // 3 # => 1 Division entière
7 % 3 # => 2 Modulo
2**4 # => 16 Exponentation
not True == False # => True (Valeurs booléeenes)
True and False or False # => True (Opérateurs logiques)
0 and 2 # => 0 (Opérateurs logiques sur des entiers)
0 == False # => False (Opérateurs logiques sur des entiers)
1 != 1 # => False (Inégalités
1 < 2 < 3 # => True Chaînage de comparaisons
None # Un objet de type None
"etc" is None # => False
None is None # => True
bool(0) # => False
bool("") # => False
[] # Liste 
() # Tuple 
{} # Dictionnaire
"" # Chaîne de caractères
\end{lstlisting}

\section*{Initialisations}

\begin{lstlisting}
i = 42
type(i) # => <class 'int'>
a = [1, 2, 3, 4] # Nouvelle instance de liste 
b = a # Référence vers la même liste
b is a # => True (b pointe sur a)
b == a # => True (les objets a et b sont égaux)
c = a[:] # Copie de la liste
c is a # => False (c n'est pas une référence vers a)
d = a.copy() # Autre manière de copier une liste
\end{lstlisting}

\section*{Chaînes de caractères}

\begin{lstlisting}
"Hello, World!"
'Hello, World, too!'
"""Multi 
Ligne"""
"Carottes " + "Cuites" # => "Carottes Cuites"
"A" "B" # => "AB"
"Trois" * 3 # => "TroisTroisTrois"
"Webb"[0] # => "W"
"{} et {}".format("Python", "Java") # => "Python et Java"
"{name} veut manger {eat}"
    .format(name="Bob", eat="pizza") # => "Bob veut manger pizza"
name = "Bob"
"%s %d" % ("Hello", 42) # => "Hello 42"
"{1} {0} {0} {1}".format("b", "a") # => "a b b a"
f"Bonjour {name}!" # => "Bonjour Bob!"
print("Hello", end="!") # => "Hello!"
print("Hello", end="") # => "Hello"
print("Hello") # => "Hello\n"
print("Error", file=sys.stderr) # => "Error" sur stderr 
\end{lstlisting}

\section*{Listes}
Une liste est un tableau dynamique d'objets
\begin{lstlisting}
a = [] # Nouvelle instance de liste vide
len(a) # => 0
a.append(1) # a = [1]
a.append(2) # a = [1, 2]
a.append(3, 4) # a = [1, 2, 3, 4]
a.pop() # a = [4], retourne 3
a[0] # => 1
a[-1] # => 3 (en partant de la fin)
a[42] # => IndexError
a[0:3] # => [1, 2, 3]
a[0:3:1] # => [1, 2, 3]
a[0:] # de 0 à la fin
a[:2] # du début à 2
a[::2] # Un élément sur 2
a[::-1] # Inverse la liste 
del a[0] # a = [2] (supprime 1)
b = [4, 5, 6]
a + b # => [1, 2, 3, 4, 5, 6]
b.extend([8, 9]) # b = [4, 5, 6, 8, 9]
1 in a # => True
a = [1, 3.15, "Hello", [1, 2]] # Liste composite
a, b, c = [1, 2, 3] # a = 1, b = 2, c = 3
a, b = b, a # a = 2, b = 1
\end{lstlisting}

\section*{Tuples}
Un tuple est une liste non modifiable.
\begin{lstlisting}
a = (1, 2, 3) # Nouvelle instance de tuple
a[0] # => 1
a[0] = 42 # TypeError
\end{lstlisting}

\section*{Dictionnaires}
Un dictionnaire est un hashmap, une collection de paires, valeurs 
\begin{lstlisting}
a = {} # Nouvelle instance de dictionnaire
a["key"] = "value"
b = {"key": "value"} # Nouvelle instance de dictionnaire
invalid = {[1,2]: "value"} # TypeError
valid = {(1,2): "value"} # Ok
list(a.keys()) # => ["key"]
list(a.values()) # => ["value"]
a['oops'] # KeyError
a.get("oops") # => None
a.get("oops", "default") # => "default"
\end{lstlisting}

\section*{Sets}
Un set est une collection de valeurs uniques, sans ordre.
\begin{lstlisting}
a = set() # Nouvelle instance de set
a.add(1) # a = {1}
a.add(2) # a = {1, 2}
a.add(2) # a = {1, 2} Déjà dans a
b = {2, 5, 7} # Nouvelle instance de set
a.union(b) # a = {1, 2, 5, 7}
a & b # a = {2} Intersection
a | b # a = {1, 2, 5, 7} Union
a - b # a = {1} Différence  
1 in a # => False
\end{lstlisting}

\section*{Structures de contrôle}
\begin{lstlisting}
x = 42
if x > 10:
    print("x est supérieur à 10")
elif x > 5:
    print("x est supérieur à 5")
else:
    print("x est inférieur ou égal à 5")

\end{lstlisting}

\end{multicols*}
\end{document}
