\documentclass{article}

\usepackage[a4paper, landscape, margin=1cm]{geometry}
\usepackage{fontspec}
\usepackage[french]{babel}
\usepackage[fontsize=6.5pt]{scrextend}
\usepackage[T1]{fontenc}
\usepackage{multicol}
\usepackage{tabularx}
\usepackage{sectsty}
\usepackage{lmodern}
\usepackage{stix}
\usepackage{listings}
\usepackage[dvipsnames]{xcolor}
\usepackage{multirow}
\usepackage{titlesec}
\usepackage{fancyvrb,cprotect}
\input{revision}

\ifdefined\tinted
  \pagecolor{SpringGreen!60}
\fi

\setlength{\parskip}{0.2em}
\setlength{\parindent}{0em}

% Highlight configuration for C programming language
\lstset{
  language=python,
  breaklines=true,
  keywordstyle=\bfseries\color{black},
  basicstyle=\ttfamily\color{black},
  emphstyle={\em \color{gray}},
  commentstyle=\color{gray}\ttfamily,
  mathescape=true,
  keepspaces=true,
  showspaces=false,
  showtabs=true,
  tabsize=3,
  columns=fullflexible,
}

% Configuration
\renewcommand{\familydefault}{\sfdefault}

\sectionfont{\fontsize{12}{15}\selectfont}
\subsectionfont{\fontsize{10}{12}\selectfont}

\allsectionsfont{\sffamily\underline}

% No pages numbering
\pagenumbering{gobble}

% Titles and paragraphs more compact
\titlespacing*{\section}{0pt}{0pt}{0pt}
\titlespacing*{\subsection}{0pt}{0pt}{0pt}

\newlength\mybaselinestretch
\mybaselinestretch=0pt plus 0.02pt\relax
\addtolength{\baselineskip}{\mybaselinestretch}

\setlength\parindent{0pt}
\setlength\tabcolsep{1.5pt}
\setlength{\columnseprule}{0.4pt}

% Macros
\newcommand{\tab}{\hspace{2em}}
\newcommand{\etc}{\small \ldots}
\newcommand{\any}{$\hzigzag$~}
\newcommand{\spc}{$\mathvisiblespace$}
\newcommand{\cd}{\lstinline}

\begin{document}

\begin{multicols*}{4}

\begin{center}
  {\Large \bf Carte de référence Python 3.x} \\
  HEIG-VD -- version \revision \ -- \revisiondate \\
\end{center}

Cette carte de référence peut être utilisée durant les travaux écrits
des cours utilisant \emph{python} à moins que le contraire soit explicitement formulé. Elle est une liste non exhaustive des possibilités du langage Python 3. Ce travail est inspiré de \emph{Learn X in Y}.

\section*{Différences entre Python 2.x et Python 3.x}

\begin{lstlisting}
print('hello') print 'hello'
2/3 == 0.66    2/3 = 0
"unicode"      u"unicode"
range(1,2)     xrange(1,2)
raise Exception("error") raise "error"
\end{lstlisting}

\section*{Opérateurs et généralités}

\begin{lstlisting}
5.0 // 3                # => 1 Division entière
7 % 3                   # => 2 Modulo
2**4                    # => 16 Exponentation
not True == False       # => True (Valeurs booléeenes)
True and False or False # => True 
0 and 2                 # => 0 (Op. logiques sur int)
0 == False              # => False 
1 != 1                  # => False (Inégalités)
1 < 2 < 3               # => True (Chaînage)
None                    # Un objet de type None
"etc" is None           # => False
None is None            # => True
bool(0)                 # => False
bool("")                # => False

[]                      # Liste 
()                      # Tuple 
{}                      # Dictionnaire
""                      # Chaîne de caractères
''                      # Chaîne de caractères
\end{lstlisting}

\section*{Initialisations}

\begin{lstlisting}
i = 42 + 2j       # Init une var complexe
type(i)           # => <class 'complex'>
a = [1, 2, 3, 4]  # Nouvelle instance de liste 
b = a             # Référence vers la même liste
b is a            # => True (b pointe sur a)
b == a            # => True (objets a et b égaux)
c = a[:]          # Copie de la liste
c is a            # => False (c pas référence sur a)
d = a.copy()      # Autre manière de copier une liste
\end{lstlisting}

\section*{Chaînes de caractères}

\begin{lstlisting}
"Hello, World!"
'Hello, World, too!'
"""Multi 
Ligne"""
"Carottes " + "Cuites" # => "Carottes Cuites"
"A" "B"                # => "AB"
"Trois" * 3            # => "TroisTroisTrois"
"Webb"[0]              # => "W"
"{} et {}".format("Python", "Java") # => "Python et Java"
"{name} veut manger {eat}"
    .format(name="Bob", eat="pizza") # => "Bob veut manger pizza"
name = "Bob"
"%s %d" % ("Hello", 42) # => "Hello 42"
"{1} {0} {0} {1}".format("b", "a") # => "a b b a"
f"Bonjour {name}!" # => "Bonjour Bob!"
print("Hello", end="!") # => "Hello!"
print("Hello", end="") # => "Hello"
print("Hello") # => "Hello\n"
print("Error", file=sys.stderr) # => "Error" sur stderr 

','.join(["a", "b", "c"]) # => "a,b,c"
"foobar".upper() # => "FOOBAR"
"FOOBAR".lower() # => "foobar"
"foobar".capitalize() # => "Foobar"
"un chat gris".title() # => "Un chat gris"

\end{lstlisting}

\section*{Listes}
Une liste est un tableau dynamique d'objets
\begin{lstlisting}
a = []              # Nouvelle instance de liste vide
len(a)              # => 0
a.append(1)         # a = [1]
a.append(2)         # a = [1, 2]
a.append(3, 4)      # a = [1, 2, 3, 4]
a.pop()             # a = [4], retourne 3
a[0]                # => 1
a[-1]               # => 3 (en partant de la fin)
a[42]               # => IndexError
a[0:3]              # => [1, 2, 3]
a[0:3:1]            # => [1, 2, 3]
a[0:]               # de 0 à la fin
a[:2]               # du début à 2
a[::2]              # Un élément sur 2
a[::-1]             # Inverse la liste
del a[0]            # a = [2] (supprime 1)
b = [4, 5, 6]
a + b                       # => [1, 2, 3, 4, 5, 6]
b.extend([8, 9])            # b = [4, 5, 6, 8, 9]
1 in a                      # => True
a = [1, 3.15, "Hu", [1, 2]] # Liste composite
a, b, c = [1, 2, 3]         # a = 1, b = 2, c = 3
a, b = b, a                 # a = 2, b = 1
\end{lstlisting}

\section*{Tuples}
Un tuple est une liste non modifiable.
\begin{lstlisting}
a = (1, 2, 3)  # Nouvelle instance de tuple
a[0]           # => 1
a[0] = 42      # TypeError
\end{lstlisting}

\section*{Dictionnaires}
Un dictionnaire est un hashmap, une collection de paires, valeurs 
\begin{lstlisting}
a = {}              # Nouvelle instance de dict
a["key"] = "value"
b = {"key": "value"}        
invalid = {[1,2]: "value"}  # TypeError
valid = {(1,2): "value"}    # Ok
list(a.keys())              # => ["key"]
list(a.values())            # => ["value"]
a['oops']                   # KeyError
a.get("oops")               # => None
a.get("oops", "default")    # => "default"
\end{lstlisting}

\section*{Sets}
Un set est une collection de valeurs uniques, sans ordre.
\begin{lstlisting}
a = set()      # Nouvelle instance de set
a.add(1)       # a = {1}
a.add(2)       # a = {1, 2}
a.add(2)       # a = {1, 2} Déjà dans a
b = {2, 5, 7}  # Nouvelle instance de set
a.union(b)     # a = {1, 2, 5, 7}
a & b          # a = {2} Intersection
a | b          # a = {1, 2, 5, 7} Union
a - b          # a = {1} Différence
1 in a         # => False
\end{lstlisting}

\section*{Structures de contrôle}
\begin{lstlisting}
x = 42
if x > 10: # Conditionnel if-elif-else
    print("x est supérieur à 10")
elif x > 5:
    print("x est supérieur à 5")
else:
    print("x est inférieur ou égal à 5")

# Boucle for
for animal in ["chat", "chien", "oiseau"]:
    print(f"Un {animal} est un mammifère")

for i in range(4): print(i, end=" ") # 0 1 2 3 4
for i in range(4, 8): print(i, end=" ") # 4 5 6 7
for i in range(0, 10, 2): print(i, end=" ") # 0 2 4 6 8

animaux = ["chat", "chien", "oiseau"]
for i, animal in enumerate(animaux):
    print(f"{i}: {animal}")
# 0: chat 1: chien 2: oiseau

d = {"c": 1, "b": 2, "a": 3}
for key in d: print(key) # c b a
for key in sorted(d): print(key) # a b c
for key, value in d.items(): print(key, value) # a 3 b 2 c 1

# Boucle while
i = 0
while i < 4: print(i += 1) # 1 2 3 4
  
# Switch (Python >=3.10)
status = 400
match status:
  case 200: print("OK")
  case 400: print("Bad Request")  
  case 404: print("Not Found")
  case _: print("Status inconnu") # Default      
\end{lstlisting}

\section*{Fonctions}
\begin{lstlisting}
def foo: 
    pass # Pass est un mot clé qui ne fait rien

def add(a, b, c): # a, b, c sont des paramètres
    return a + b + c

def add(a, b, c=0): # Paramètre optionnel
    return a + b + c

def add(a, b, *args): # Paramètre de type tuple
    return a + b + sum(args)

def display(*args): 
    for arg in args: print(arg) \# Affiche toutes les valeurs

def display(**kwargs): \# Paramètre de type dictionnaire
    for key, value in kwargs.items(): print(key, value)

def bar(a, b=0, *args, **kwargs):
    print(a, b, args, kwargs) 
bar(1, 2, 3, c=9, d=(1,2)) \# => 1 2 (3,) {'c': 9, 'd': (1, 2)}

# Fonctions anonymes
f = lambda x, y: x + y
f(1, 2) # => 3
def iterate(x, f):
    for i in range(x):
        f(i)
iterate(3, lambda i: print(i)) # => 0 1 2

(lambda x: x > 2)(3) # => True
(lambda x, y: x ** 2 + y ** 2)(2, 3) # => 5

x = 0 # Scope global (mauvaise pratique)
def foo():
    global x
    x = 42
foo() # x => 42

def foo(): # Fonctions imbriquées
    def bar():
        print("bar")
    bar()

def double_numbers(iterable): # Générateur
for i in iterable:
    yield i + i

# Annotation de fonction (Python >=3.8)
def add(x: int, y: int) -> int:
    return x + y
\end{lstlisting}

\section*{Fichiers}
\begin{lstlisting}
# Ouverture d'un fichier
with open("file.txt", "r") as fp:
    for line in fp:
        print(line) # Affiche chaque ligne
# r: lecture seule, w: écriture seule, a: écriture à la suite
# b, mode binaire (rb, wb, ...)
# t, mode texte (rt, wt, ...)
# +, mode plus (r+, w+, ...)
# U, mode UTF-8 (rU, wU, ...)
\end{lstlisting}

\section*{Exceptions}
\begin{lstlisting}
try: 
    # Code susceptible d'échouer
except Exception as e: 
    # Code exécuté en cas d'erreur
else: 
    # Code exécuté en cas de succès
finally: 
    # Code exécuté quelque soit le résultat
\end{lstlisting}

\section*{Compréhensions}
\begin{lstlisting}
[2 ** x for x in range(5)] # => [1, 2, 4, 8, 16]
{x for x in range(5)} # => {0, 1, 2, 3, 4}
{x: x ** 2 for x in range(5)} # => {0: 0, 1: 1, 2: 4, 3: 9, 4: 16}
[x for x in range(5) if x % 2 == 0] # => [0, 2, 4]

# Générateur sous forme de compréhension      
values = (-x for x in [1,2,3,4,5])
\end{lstlisting}

\section*{Math}
\begin{lstlisting}
import math
print(math.sqrt(16)) # => 4.0
print(math.sin(math.pi / 2)) # => 1.0
dir(math) # affiche les méthodes disponibles
\end{lstlisting}

\section*{Classes}
\begin{lstlisting}
class Human:   
    # Attribut de classe partagé par toutes les instances
    species = "Homo Sapiens" 

    # Initialiseur (appelé à l'instanciation)
    def __init__(self, name):
        self.name = name # Propriété publique
        self._age = 0 # Propriété cachée
    
    def say(self, msg):
        print(f"{self.name}: {msg}")

    def sing(self):
        return 'yo... yo... microphone check... one two... one two...'
        
    # Partagée par toutes les instances
    @classmethod
    def get_species(cls):
        return cls.species

    # Méthode sans références sur la classe ou l'instance
    @staticmethod
    def grunt():
        print("*grunt*")

    # Getter
    @property
    def age(self):
        return self._age

    # Setter 
    @age.setter
    def age(self, value):
        if value < 0:
            raise ValueError("Age non valide")
        self._age = value

    # Suppresseur 
    @age.deleter
    def age(self):
        raise AttributeError("Age non supprimable")

    # Methodes magiques
    def __str__(self):
        return f"{self.name} ({self.age})"
    
    def __repr__(self):
        return f"Human('{self.name}', {self.age})"
    
    def __len__(self):
        return len(self.name)

# Héritage
class Supehero(Human):
    species = "Super Homme"

    def __init__(self, name, movie=False,
        superpowers=["super strength", "bulletproofing"]):
        self.fictional = True 
        self.movie = movie
        self.superpowers = superpowers
        super().__init__(name) # Appel du constructeur de la classe parente 

    def sing(self): # Surcharge de sing
        return 'Dun, dun, DUN!'

# Héritage multiple 
class Batman(Human, Superhero):
    def __init__(self, *args, **kwargs):
        Superhero.__init__(self, 'anonymous', movie=True,
        superpowers=['Wealthy'], *args, **kwargs)
        Bat.__init__(self, *args, can_fly=False, **kwargs)

        # override the value for the name attribute
        self.name = 'Sad Affleck'
\end{lstlisting}

\section*{Décorateurs}
\begin{lstlisting}
from functools import wraps

def beg(target_function):
    @wraps(target_function)
    def wrapper(*args, **kwargs):
        msg, say_please = target_function(*args, **kwargs)
        if say_please:
            return "{} {}".format(msg, "Please! I am poor :(")
        return msg

    return wrapper

@beg
def say(say_please=False):
    msg = "Can you buy me a beer?"
    return msg, say_please

print(say())                 # Can you buy me a beer?
print(say(say_please=True))  # Can you buy me a beer? Please! I am poor :(
\end{lstlisting}

\section*{Loggeur et traceurs}
\begin{lstlisting}
import logging
logging.warning('Watch out!')  # Message sur la console
logging.info('I told you so')  # N'affiche rien 

logging.basicConfig(
    filename='example.log', encoding='utf-8', level=logging.DEBUG)
logging.debug('Un message')

logger = logging.getLogger(__name__) # Création d'un logger
logger.setLevel(logging.DEBUG) 

formatter = logging.Formatter(
  '%(asctime)s - %(name)s - %(levelname)s - %(message)s')
logger.setFormatter(formatter)
logger.addHandler(logging.StreamHandler()) # Affiche sur la console

\end{lstlisting}

\section*{Singleton}
\begin{lstlisting}
class Singleton(type):
_instances = {}
def __call__(cls, *args, **kwargs):
    if cls not in cls._instances:
        cls._instances[cls] = super(Singleton, cls).__call__(*args, **kwargs)
    return cls._instances[cls]

class MyClass(BaseClass, metaclass=Singleton):
    pass
\end{lstlisting}

\section*{Import et modules}
\begin{lstlisting}
from math import sqrt 
from math import *     # Mauvaise pratique

# Comportement par défaut à l'appel du script
if __name__ == '__main__':
    main()
\end{lstlisting}

\section*{Modules}
\begin{description}
\item[beautifulsoup4] bibliothèque pour manipuler des pages web
\item[flask] framework pour créer des applications web
\item[functools] bibliothèque pour manipuler des fonctions
\item[itertools] bibliothèque pour manipuler des itérateurs
\item[jinja] moteur de template
\item[matplotlib] bibliothèque pour créer des graphiques
\item[numpy] bibliothèque pour manipuler des tableaux
\item[opencv] bibliothèque pour manipuler des images
\item[pandas] bibliothèque pour manipuler des données
\item[pillow] bibliothèque pour manipuler des images
\item[pint] bibliothèque pour manipuler des unités
\item[pytest] bibliothèque pour tester des applications 
\item[requests] bibliothèque pour manipuler des requêtes HTTP
\item[scipy] bibliothèque pour manipuler des fonctions mathématiques
\item[seaborn] bibliothèque pour manipuler des graphiques
\item[yaml] bibliothèque pour manipuler des fichiers YAML
\item[json] bibliothèque pour manipuler des fichiers JSON
\end{description}

\subsection*{Numpy}
\begin{lstlisting}
>>> import numpy as np
>>> a = np.arange(15).reshape(3, 5)
>>> a
array([[ 0,  1,  2,  3,  4],
       [ 5,  6,  7,  8,  9],
       [10, 11, 12, 13, 14]])
>>> a.shape
(3, 5)
>>> a.ndim
2
>>> a.dtype.name
'int64'
>>> a.itemsize
8
>>> a.size
15
>>> type(a)
<class 'numpy.ndarray'>
>>> np.array([6, 7, 8]) * 2
array([12, 14, 16])
>>> np.zeros((3, 5))
>>> np.zeros((3, 5), dtype=np.int16)
array([[0, 0, 0, 0, 0],
       [0, 0, 0, 0, 0],
       [0, 0, 0, 0, 0]])
>>> np.array([[1, 2], [3, 4]], dtype=complex)
array([[1.+0.j, 2.+0.j],
       [3.+0.j, 4.+0.j]])
>>> np.linspace(0, 2, 9)
array([0. , 0.2, 0.4, 0.6, 0.8, 1. , 1.2, 1.4, 1.6])
\end{lstlisting}

\subsection*{Matplotlib}
\begin{lstlisting}
>>> import matplotlib.pyplot as plt
>>> x = np.linspace(0, 2, 100)
>>> y = np.sin(2 * np.pi * x)
>>> plt.plot(x, y)
[<matplotlib.lines.Line2D object at 0x7f8b8b8b9b90>]
>>> plt.ylabel('some numbers')
>>> plt.show()
\end{lstlisting}

\subsection*{Functools}
\begin{lstlisting}
import functools 
@cache
def factorial(n):
    return n * factorial(n-1) if n else 1
\end{lstlisting}

\subsection*{Collections}
\begin{lstlisting}
from collections.abc import Mapping
class DefaultDict(Mapping):
    def __init__(self, default_factory=None, *args, **kwargs):
        self.default_factory = default_factory
        self.store = dict()
        self.update(dict(*args, **kwargs))
    def __getitem__(self, key):
        if key not in self.store:
            self.store[key] = self.default_factory()
        return self.store[key]
    def __iter__(self):
        return iter(self.store)
    def __len__(self):
        return len(self.store)
    def __repr__(self):
        return 'DefaultDict(%s, %s)' % (self.default_factory, self.store)
\end{lstlisting}

\subsection*{Expressions régulières}
\begin{lstlisting}
import re 
re.findall(r'\bf[a-z]*', 'which foot or hand fell fastest')
['foot', 'fell', 'fastest']

\end{lstlisting}


\end{multicols*}

\end{document}
