\documentclass{article}

\usepackage[a4paper, landscape, margin=1cm]{geometry}
\usepackage{fontspec}
\usepackage[french]{babel}
\usepackage[fontsize=6.5pt]{scrextend}
\usepackage[T1]{fontenc}
\usepackage{multicol}
\usepackage{tabularx}
\usepackage{sectsty}
\usepackage{lmodern}
\usepackage{stix}
\usepackage{listings}
\usepackage{xcolor}
\usepackage{multirow}
\usepackage{titlesec}

\input{revision}

\setlength{\parskip}{0.2em}
\setlength{\parindent}{0em}

% Highlight configuration for C programming language
\lstset{
  language=c,
  breaklines=true,
  keywordstyle=\bfseries\color{black},
  basicstyle=\ttfamily\color{black},
  emphstyle={\em \color{gray}},
  emph={expr, type, NAME, ptr, name, expr, value, filename, label, member, type},
  mathescape=true,
  keepspaces=true,
  showspaces=false,
  showtabs=true,
  tabsize=3,
  columns=fullflexible,
  escapeinside={(*}{*)}
}

% Configuration
\renewcommand{\familydefault}{\sfdefault}

\sectionfont{\fontsize{12}{15}\selectfont}
\subsectionfont{\fontsize{10}{12}\selectfont}

\allsectionsfont{\sffamily\underline}

% No pages numbering
\pagenumbering{gobble}

% Titles and paragraphs more compact
\titlespacing*{\section}{0pt}{1ex plus .2ex minus 0.1ex}{1ex plus .2ex minus 0.1ex}
\titlespacing*{\subsection}{0pt}{1ex plus .2ex minus 0.1ex}{1ex plus .2ex minus 0.1ex}

\newlength\mybaselinestretch
\mybaselinestretch=0pt plus 0.02pt\relax
\addtolength{\baselineskip}{\mybaselinestretch}

\setlength\parindent{0pt}
\setlength\tabcolsep{1.5pt}
\setlength{\columnseprule}{0.4pt}

% Macros
\newcommand{\tab}{\hspace{2em}}
\newcommand{\etc}{\small \ldots}
\newcommand{\any}{$\hzigzag$~}
\newcommand{\spc}{$\mathvisiblespace$}
\newcommand{\cd}{\lstinline}

\begin{document}

\begin{multicols*}{3}

\begin{center}
  {\Large \bf Carte de référence C++11/C++17} \\
  HEIG-VD -- version \revision \ -- \revisiondate \\
\end{center}

Cette carte de référence peut être utilisée durant les travaux écrits et examens
des cours \emph{progoo} et \emph{progoox} à moins que le contraire soit explicitement formulé.
Elle est une liste non exhaustive des possibilités du langage C++. Elle est un complément à la carte
de référence du langage C.

Ce travail est inspiré de \emph{Learn X in Y} et de la carte de référence de Matt Mahoney.

Signification des termes utilisés dans cette carte de référence.

\begin{tabularx}{\linewidth}{
  >{\hsize=0.5\hsize}X% 10% of 4\hsize
  >{\hsize=1.5\hsize}X% 30% of 4\hsize
  >{\hsize=0.5\hsize}X% 30% of 4\hsize
  >{\hsize=1.5\hsize}X% 30% of 4\hsize
     % sum=4.0\hsize for 4 columns
  }

  \tt \etc      & Continuation logique    & \tt \any    & N'importe quoi d'accepté \\
  \tt /\any/    & Expression régulière    & \tt \spc    & Espace obligatoire \\
  \cd{type}     & \tt int, long, float, ... & \cd{name} & \tt /[A-Za-z][A-Za-z0-9\_]+/ \\
  \cd{value}    & Valeur & \cd{NAME} & \tt /[A-Z][A-Z0-9\_]+/ \\
  \cd{filename} & Chemin de fichier relatif & \cd{expr}   & e.g. \tt a + b \\
\end{tabularx}
\hrule

\section*{Généralités}

\begin{lstlisting}
for(auto x; iterable)   // Boucle itérative sur itérable
int& r = x;             // Référence r sur x
\end{lstlisting}

\section*{Pointeurs et références}

\section*{Templates}

\begin{lstlisting}
template <class T> T f(T t);            // Surcharge f pour tous les types
template <class T> class X {            // Classe avec paramètre T
   X(T t); };                           // Constructeur
template <class T> X<T>::X(T t) {}      // Définition du constructeur
X<int> x(3);                            // Un objet de type X de int
template <class T, class U=T, int n=0>  // Template avec paramètres par défaults
\end{lstlisting}

\section*{Namespaces}

\begin{lstlisting}
extern "C" { void f(); }    // Si f() a été compilé dans un objet C
namespace N { class T {}; } // Cache la classe T dans le namespace N
N::T t;                     // Utilise le nom T du namespace N
using namespace N;          // Rends T visible sans préfix N::

using namespace std;        // Permet les appels standards sans std::
\end{lstlisting}

\section*{Exceptions}

\begin{lstlisting}
try { a; }              // Essai d'exécution
catch (T t) { b; }      // Capture si exception levée
catch (...) { c; }      // Autre capture si exception non capturée
\end{lstlisting}

\section*{Classes}

\begin{lstlisting}
class T {                    // Déclaration nouvelle classe
private:                     // Accès seulement aux membres de T
protected:                   // Accès possible aux classes dérivées
public:                      // Accès à tous
   int x;                    // Attribut de classe
   void f();                 // Méthode de classe
   int g(void) { return 1; } // Méthode déclarées
   T(): x(1) {}              // Constructeur avec liste d'initialisation
   ~T();                     // Déstructeur
}

void T::f() {                // Déclaration méthode f, membre de T
   this->x = x;              // Adresse de l'instance
}

int T::y = 2;                // Initialisation d'attribut statique

T::l();                      // Appel de méthode statique
\end{lstlisting}

\section*{Polymorphisme}

\begin{lstlisting}
T operator+(T x, T y);  // Surcharge opérateur addition de type T
T operator-(T x);       // Surcharge opérateur unaire de negation -x
T operator++(int);      // Surcharge opérateur postfix
\end{lstlisting}

\section*{STL}
\subsection*{Vector}

\subsection*{Tuples}

\begin{lstlisting}
#include <tuple>

auto x = make_tuple(10, 'A');
const int maxN = 1e9;
const int maxL = 15;
auto y = make_tuple(maxN, maxL);

// Affichage
cout << get<0>(x) << " " << get<1>(x) << "\n"; // Affiche "10 A"
cout << get<0>(y) << " " << get<1>(y) << "\n"; // Affiche "1000000000 15"

// Unpacking
int i;
int c;
tie(i, c) = x;

tuple<int, char, double> z(11, 'A', 3.1415);

cout << tuple_size<decltype(z)>::value << "\n"; // Affiche 3

auto concatenated = tuple_cat(x, y, z); // (10, 'A', 1e9, 15, 11, 'A', 3.1415)

\end{lstlisting}

\subsection*{Set}

\begin{lstlisting}
#include <set>
set<int> ST;
ST.insert(4);
ST.insert(8);
ST.insert(15);
ST.erase(8);
set<int>::iterator it;
for(it=ST.begin();it<ST.end();it++) {
    cout << *it << endl;
}
\end{lstlisting}

\section*{Fonctions anonymes (lambda)}

\begin{lstlisting}
vector<pair<int, int> > tester;
tester.push_back(make_pair(3,6));
tester.push_back(make_pair(1,9));
tester.push_back(make_pair(5,0));
sort(
    tester.begin(),
    tester.end(),
    [](const pair<int, int>& lhs, const pair<int, int>& rhs) {
        return lhs.second < rhs.second;
    }
);

struct S { void f(int i); };
void S::f(int i) {
    [&, i]{};      // Ok
    [&, &i]{};     // Erreur : i précédé par & quand & est le défaut
    [=, this]{};   // Erreur : this quand = est défaut
    [=, *this]{ }; // OK
    [i, i]{};      // Erreur : i répété
}
\end{lstlisting}

\section*{Tokens alternatifs}

\begin{lstlisting}
true and false          // Conjonction logique
true or false           // Disjonction logique
not true                // Négation logique

compl 4
4 bitor 3               // Disjonction bit à bit
4 bitand 3              // Conjonction bit à bit
4 xor 3                 // Disjonction exclusive bit à bit
\end{lstlisting}

\section*{Étrangetés}

\begin{lstlisting}
// Surcharge de méthode privées
class Foo {
private:
   virtual void bar();
};
class FooSub : public Foo {
   virtual void bar();      // Surcharge Foo::bar!
};

assert(0 == false and false == null); // La plupart du temps



\end{lstlisting}

\end{multicols*}
\end{document}
